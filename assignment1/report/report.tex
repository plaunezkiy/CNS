\documentclass{article}

\usepackage[english]{babel}
\usepackage{amssymb}
\usepackage{amsmath}


\begin{document}

\title{CNS Assignment: Inhibitory-Stabilized Networks}
\author{B204511}
\date{November 2024}
\maketitle

\section{Introduction}
Introduce ISN - explain the model and its components

Inhibitory neurons maintain the network activity in a balanced
regime, preventing endless excitation, preventing the network from 
being functional and balanced. This is achieved by the
inhibitory neurons being activated by the excitatory neurons, to which they
are connected via synapses.
The ISN neurons can be described by the following 2 equations:

\begin{equation}
    \tau_E \frac{dV_E}{dt} = 
    -(V_E - V_{\text{rest}}) 
    + W_{EE} \varphi(V_E) 
    - W_{EI} \varphi(V_I) + u_E
\end{equation}

\begin{equation}
    \tau_I \frac{dV_I}{dt} = 
    -(V_I - V_{\text{rest}}) 
    + W_{IE} \varphi(V_E) 
    - W_{II} \varphi(V_I) + u_I
\end{equation}

They describe Excitatory (E) and Inhibitory (I) neuron behaviour.
To understand how the 2 come together, it is important to understand 
the underlying model of a nueron and how the interaction is introduced.

\subsection{Leaky Integrate-and-Fire Neuron}
\{equation\}

Breakdown:
1. change in membrane potential is modelled as an ODE 
2. Tau - time for a signal to fully be absorbed
3. Resting potential
4. Conductance
5. External current

External current is the component of interest - it allows
to establish a communcation channel between the nuerons - 
the \textbf{total} synaptic current of the adjacent neurons 
is the external input to each indivudual neuron.

\{equations for modelling synaptic current \}

\subsection{Considerations}
1. The model keeps the key mechanisms of a real neural network - oscillations, 
reaction to external input, delayed reaction, inter-neural interactions

2. While capturing important features, the model abstracts away the 
full detail of chemical \& biological structure of the cells and the medium around them 
to spare the  complexity in order to make the model cheaper computationally and more
tractable analytically. 

2. The "transfer function" is linear, whereas in reality
it is driven by a stochastically opening and closing ion-channels, which would
have a different shape

3. The model is a lot more computationally affordable - we can run large scale simulations
on 1 ODE per Neuron, instead of 4 as it is with the biologically more plausible Hodgkin-Huxley model

4. Simpler model also is more amenable to theoretic analysis - helps to understand
network stability, activity patterns and interpret them.



\section{Problem 1}

\section{Problem 2}

\section{Problem 3}

\end{document}


